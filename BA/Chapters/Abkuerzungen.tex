\chapter*{Nomenklatur}
\begin{onehalfspacing}
\begin{longtable}[h]{p{0.15\textwidth} p{0.65\textwidth} p{0.1\textwidth}}
		\caption*{\textbf{Formelzeichen und Einheiten}} \\
		\\
		\textbf{Symbol} & \textbf{Bedeutung} & \textbf{Einheit} \\ %\hline 
		\endhead
		\\
		\multicolumn{3}{c}{Fortsetzung auf der nächsten Seite} \\
		\endfoot
		\multicolumn{3}{c}{ } \\
		\endlastfoot
		
		$A$ & Fläche & \squaremetre\\
		$c_{p}$&spezifische Wärmekapazität bei konstantem Druck&\joule\per(\kilogram\usk\kelvin)\\
			$C$&Wärmekapazität&\watt\per\kilogram\\
		$H $ & Enthalpie & \joule\\	
		$\dot{M}$ & Massenstrom & \kilogram\per\second\\	
		$\dot{m}$ & spezifischer Massenstrom & \kilogram\per(\second\usk\metre)\\
		$p$ & Druck & \pascal\\
		$\dot{Q}$ & Wärmestrom & \watt\\
		$R$ & spezifische Gaskonstante & \joule\per(\kilogram\usk\kelvin)\\
		$S$ & Entropie & \joule\per\kelvin\\
		$T$ & Temperatur & \kelvin\\
		$t$ & Zeit & \second\\
		$U_{T}$ & Wärmedurchgangskoeffizient & \watt\per(\kilogram\usk\kelvin)\\
		$h$ & Wärmeübergangskoeffizient & \watt\per(\squaremetre\usk\kelvin)\\		
		$V$ & Volumen & \cubic\meter\\
		$\dot{V}$&Volumenstrom&\cubic\meter\per\second\\
		$P$ & Leistung & \watt\\
		$Y$ & Wasserbeladung der Luft & \gram\per\kilogram\\
		
		
\end{longtable}

\begin{longtable}[h]{p{0.15\textwidth} p{0.65\textwidth} p{0.1\textwidth}}
		\caption*{\textbf{Griechische Formelzeichen}} \\
		\\
		\textbf{Symbol} & \textbf{Bedeutung} & \textbf{Einheit} \\ %\hline 
		\endhead
		\\
		\multicolumn{3}{c}{Fortsetzung auf der nächsten Seite} \\
		\endfoot
		\multicolumn{3}{c}{ } \\
		\endlastfoot
		
		$\eta_{C}$ & Carnot-Wirkungsgrad & ---\\
		$\kappa_{\mathrm{E}}$ & exergetische Aufwandszahl der Wärmeerzeugung & ---\\
		$\kappa_{\mathrm{T}}$ & exergetische Aufwandszahl des Wärmetransfers & ---\\
		$\Phi$ & thermische Leistung & \watt\\
		$\varrho$& Massendichte&\kilogrampercubicmetre\\
			$\sigma$&Temperaturspreizung&\kelvin\\
		$\vartheta $ & Temperatur  & \degreecelsius\\
		$\Delta\vartheta $ & Temperaturdifferenz  &\kelvin\\
		
\end{longtable}

\begin{longtable}[h]{p{0.15\textwidth} p{0.75\textwidth}}
		\caption*{\textbf{Indizes und Abkürzungen}} \\
		\\
		\textbf{Symbol} & \textbf{Bedeutung} \\ %\hline 
		\endhead
		\\
		\multicolumn{2}{c}{Fortsetzung auf der nächsten Seite} \\
		\endfoot
		\multicolumn{2}{c}{ } \\
		\endlastfoot
		
		0 & Referenzzustand (\emph{ambient dead state})\\
		Al & Aluminium\\ 
		aus & Ausgang\\		
		Cu & Kupfer \\
		cd & Verflüssiger (\emph{engl.: condensor})\\
		cp & Verdichter (\emph{engl.: compressor})\\
		D & Dampf \\     
		EES & Engineering Equation Solver \\
		ein & Eingang \\
		ev & Verdampfer (\emph{engl.: evaporator})\\
		F & Flüssigkeit\\	
		K & (Kälte-)Kreis\\
		KM & Kältemittel\\
		LabView & Programmiersprache und Entwicklungsumgebung für die Messdatenerfassung der Firma National Instruments\\
		L & Luft\\
		max & Maximum \\
		min & Minimum \\
		NTU & Anzahl der Übertragungseinheiten (\emph{engl.: Number of transfer units}) \\
		P & Durchgang (\emph{engl.: Pass})\\
		R & Verhältnis (\emph{engl.: ratio})\\
		SH & Überhitzung (\emph{engl.: Superheat})\\

		
\end{longtable}
\end{onehalfspacing}
