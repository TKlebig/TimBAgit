It is necessary to reduce CO2 and greenhouse gas emissions in order to limit the global increase in temperature and to avoid undesirable effects on the climate. The EU's F-gas regulation pursues these objectives. With a share of \unit{14}{\%} of the electricity energy demand and about \unit{5}{\%} of direct and indirect greenhouse gas emissions, the refrigeration technology in Germany offers great saving potential here. Switching from synthetic (high GWP) to natural refrigerants (low GWP) is a sustainable starting point.
There is great potential as a substitute refrigerant for propane (R290) in refrigeration circuits. The maximum capacity in cooling circuits is restricted to \unit{150}{\gram} with regard to safety standards.
Due to insufficient product cooling of a refrigerated appliance converted to propane, experimental studies are carried out in a climatic chamber under constant conditions and changing favorable parameters and components. This serves to identify the causes of the insufficient cooling and to reduce the refrigerant capacity of a refrigeration circuit while maximizing its cooling capacity.
In the bachelor thesis investigations with different engines and refrigerant oils are carried out. Comparisons of test data are evaluated according to the standard and compared with regard to the system behavior. From this, potential for change is derived.
By connecting the evaporator tubes in a modified order, a promising performance gain is achieved. For the refrigeration circuit investigation of this influence, a mathematical model of the evaporator is developed and validated on the basis of experimental data.
The knowledge gained from this is used to identify the optimum operating point of the system with regard to component selection and parameters.
The work is completed with summary and outlook.

