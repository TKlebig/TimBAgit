Zur Begrenzung des globalen Temperaturanstiegs und zur Vermeidung unerwünschter Klimaauswirkungen ist es notwendig, CO2- sowie Treibhausgasemissionen zu reduzieren. Im Rahmen der F-Gas-Verordnung der EU werden diese Ziele verfolgt. Mit einem Anteil von \unit{14}{\%} am Elektroenergiebedarf und etwa \unit{5}{\%} an direkten und indirekten Treibhausgasemissionen bietet die Kältetechnik in Deutschland hier ein großes Einsparpotential. Ein Umstieg auf Kältemittel mit einem niedrigen Ozonabbaupotential ist ein nachhaltiger Ansatzpunkt. \newline
Großes Potential als Ersatzkältemittel in Kältekreisen bietet Propan (R290). Die maximale Füllmenge in Kältekreisen ist bzgl. Sicherheitsnormen auf \unit{150}{\gram} beschränkt. 
Wegen unzureichender Produktkühlung eines auf Propan umgerüsteten Kühlmöbels werden in einer Klimakammer experimentelle Untersuchungen durchgeführt. Unter konstanten Bedingungen und unter Veränderung günstiger Parameter sowie Komponenten sollen Maßnahmen identifiziert und durchgeführt werden, die die Kälteleistung eines Kältekreises erhöhen. Infolgedessen lässt sich die Füllmenge eines Kreises verringern. Außerdem sollen Ursachen für die unzureichende Kühlung identifiziert werden. \newline
Im Rahmen der Bachelorarbeit werden Untersuchungen mit unterschiedlichen Verdichtermotoren und Kältemittelölen durchgeführt. Vergleiche von Versuchsdaten werden nach Norm ausgewertet und hinsichtlich des Systemverhaltens gegenübergestellt. Daraus werden Veränderungspotentiale abgeleitet.
Durch die Verschaltung der Verdampferrohre in einer modifizierten Reihenfolge könnte ein vielversprechender Leistungsgewinn erzielt werden. Zur Untersuchung dieses Einflusses wird ein mathematisches Modell des Verdampfers entwickelt und anhand von Versuchsdaten validiert.
Die daraus gewonnenen Erkenntnisse werden dazu benutzt, den optimalen Betriebspunkt der Anlage hinsichtlich Komponentenauswahl und Parameter zu identifizieren. \newline
Alle im Rahmen der Bachelorarbeit betrachteten Maßnahmen erweisen sich als kälteleistungssteigernd. Eine Kombination dieser Maßnahmen resultiert in einer höheren Kälteleistung, einer tieferen Produkttemperatur und einer besseren Anlageneffizienz.
Die Arbeit wird mit einer Zusammenfassung und einem Ausblick abgeschlossen.
