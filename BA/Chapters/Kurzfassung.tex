Zur Begrenzung des globalen Temperaturanstiegs und zur Vermeidung unerwünschter Klimaauswirkungen ist es notwendig, CO2- sowie Treibhausgasemissionen zu reduzieren. Im Rahmen der F-Gas-Verordnung der EU werden diese Ziele verfolgt. Mit einem Anteil von \unit{14}{\%} am Elektroenergiebedarf und etwa \unit{5}{\%} an direkten und indirekten Treibhausgasemissionen bietet die Kältetechnik in Deutschland hier ein großes Einsparpotential. Ein Umstieg von synthetischen (GWP hoch) auf natürliche Kältemittel (GWP niedrig) ist ein nachhaltiger Ansatzpunkt. 
Großes Potential als Ersatzkältemittel in Kältekreisen bietet Propan (R290). Die maximale Füllmenge in Kältekreisen ist bzgl. Sicherheitsnormen auf \unit{150}{\gram} beschränkt. 
Wegen unzureichender Produktkühlung eines auf Propan umgerüsteten Kühlmöbels, werden in einer Klimakammer experimentelle Untersuchungen unter konstanten Bedingungen und unter Veränderung günstiger Parameter sowie Komponenten durchgeführt. Dies dient dem Ziel die Ursachen für die unzureichende Kühlung zu identifizieren sowie die Füllmenge eines Kältekreises zu verringern und gleichzeitig dessen Kälteleistung zu maximieren.
Im Rahmen der Bachelorarbeit werden Untersuchungen mit unterschiedlichen Motoren und Kältemittelölen durchgeführt. Vergleiche von Versuchsdaten werden nach Norm ausgewertet und hinsichtlich des Systemverhaltens gegenübergestellt. Daraus werden Veränderungspotentiale abgeleitet.
Durch die Verschaltung der Verdampferrohre in einer modifizierten Reihenfolge wird ein vielversprechender Leistungsgewinn erzielt. Zur Kältekreisuntersuchung dieses Einflusses wird ein mathematisches Modell des Verdampfers entwickelt und anhand von Versuchsdaten validiert.
Die daraus gewonnenen Erkenntnisse werden dazu benutzt den optimalen Betriebspunkt der Anlage hinsichtlich Komponentenauswahl und Parameter zu identifizieren.
Die Arbeit wird mit Zusammenfassung und Ausblick abgeschlossen.
