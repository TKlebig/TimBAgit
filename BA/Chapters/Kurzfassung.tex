Zur Begrenzung des globalen Temperaturanstiegs und zur Vermeidung unerwünschter Klimaauswirkungen zu ist es notwendig, CO2- sowie Treibhausgasemissionen zu reduzieren. Im Rahmen der F-Gas-Verordnung der EU werden diese Ziele verfolgt. Mit einem Anteil von \unit{14}{\%}  am Elektronenergiebedarf und etwa \unit{5}{\%} an direkten und indirekten Treibhausgasemissionen bietet die Kältetechnik in Deutschland hier ein großes Einsparpotential. Ein Umstieg von synthetischen (GWP hoch) auf natürliche Kältemittel (GWP niedrig) ist ein nachhaltiger Ansatzpunkt. 
Großes Potential als Ersatzkältemittel bietet  Propan (R290) in Kältekreisen. Die maximale Füllmenge in Kältekreisen ist bzgl. Sicherheitsnormen auf \unit{150}{\gram} beschränkt. 
Mit dem Ziel die Füllmenge eines Kältekreises zu verringern und gleichzeitig Kälteleistung zu maximieren, werden in einer Klimakammer experimentelle Untersuchungen unter konstanten Bedingungen und unter Veränderung günstiger Parameter sowie Komponenten durchgeführt.
Im Rahmen der Bachelorarbeit werden Untersuchungen mit unterschiedlichen Motoren und Kältemittelölen durchgeführt. Vergleiche von Versuchsdaten sollen nach Norm ausgewerten werden und hinsichtlich des Systemverhaltens gegenübergestellt werden. Daraus können Veränderungspotentiale abgeleitet werden.
Durch die Verschaltung der Verdampferrohre in einer modifizierten Reihenfolge wird ein vielversprechender Leistungsgewinn erzielt. Zur Kältekreisuntersuchung dieses Einflusses wird ein mathematisches Modell des Verdampfers entwickelt und anhand von Versuchsdaten validiert.
Die daraus gewonnenen Erkenntnisse werden dazu benutzt den optimalen Betriebspunkt der Anlage hinsichtlich Komponentenauswahl und Parameter zu identifizieren.
Die Arbeit wird mit Zusammenfassung und Ausblick abgeschlossen.
