\chapter{Motivation}
\label{cha:Motivation}

Um den globalen Temperaturanstieg zu begrenzen und unerwünschte Klimaauswirkungen zu vermeiden ist es notwendig CO2- sowie Treibhausgasemissionen zu reduzieren. Im Rahmen der F-Gas-Verordnung der EU werden diese Ziele verfolgt. Mit einem Anteil von 14 \%  am Elektronenergiebedarf und etwa 5 \% an direkten und indirekten Treibhausgasemissionen bietet die Kältetechnik in Deutschland großes Einsparpotential. Somit ist der Umstieg von synthetischen auf natürliche Kältemittel ein wichtiger Ansatzpunkt. 
Großes Potential als Ersatzkältemittel bietet der Einsatz von Propan (R290) in Kältekreisen, dessen maximale Füllmenge in Kältekreisen vor dem Hintergrund von Sicherheitsnormen auf 150 Gramm beschränkt ist. 

Propan Umrüstung
Reduzierung des inneren Volumens
Emerson Prüfstand AHT Möbel zur Verbesserung der Kälteleistung
Klimakammer
Zusammenstellung eines Pakets an Komponenten
Quellen:
ECODESIGN
EN1127







\section{Sektion 1}
\label{sec:Sektion 1}

\subsection{Abschnitt 1}
\label{subsec:Abschnitt1}


\chapter{Prüfstand}
\label{cha:Prüfstand}


Grafiken:
Klimakammer Inkscape
Quellen:
DIN EN ISO 23953

\chapter{Versuchsdurchführung}
\label{cha:Versuchsdurchführung}

Aus den Versuchsdaten wissen wir, dass nur die letzte Rohrstrecke überhitzt ist.

\chapter{Analyse der Messergebnisse}
\label{cha:Analyse der Messergebnisse}

\chapter{Zusammenfassung}
\label{cha:Zusammenfassung}
Verweis auf Sektion: (siehe \ref{subsec:Abschnitt1})

\begin{equation}
\dot{Q}=\dot{m}\cdot c_p \cdot \Delta \vartheta
\end{equation}



\begin{table} %Tabellen und Abbildungen sind FLOATING(flexible) Objekte. Eine Fixierung mit [htb] macht keine Sinn
	\centering
	\caption{Das ist eine Testtabelle}\vspace{6pt}
	\label{tab:Table}
\begin{tabular}{cr} 
\toprule
\textbf{Linke Spalte} & \textbf{Rechte Spalte}\\
\midrule 
oben links & oben rechts \\ \midrule 
unten links & unten rechts \\ 
\bottomrule 
\end{tabular}
\end{table}

\Blindtext

Reference to figure\,\ref{fig:RWTH-Aachen-Logo}
\begin{figure} %Tabellen und Abbildungen sind FLOATING(flexible) Objekte. Eine Fixierung mit [htb] macht keine Sinn
\centering
\includegraphics[width=0.60\textwidth]{Pictures/RWTHAACHENUNIVERSITY_sw.png}
\caption{RWTH-Aachen-Logo}
\label{fig:RWTH-Aachen-Logo}
\end{figure}

\Blindtext
\begin{figure}
\includegraphics[scale=1]{Pictures/halbeSeite1}
\caption{Erste halbseitige Grafik}
\label{fig:sinnvolles_label1}
\end{figure}

\begin{figure}
\includegraphics[scale=1]{Pictures/halbeSeite2}
\caption{zweite halbseitige Grafik}
\label{fig:sinnvolles_label2}
\end{figure}
\Blindtext

\begin{figure}
\includegraphics[scale=1]{Pictures/halbeSeite1}
\caption{Unter eine halbseitige Grafik passt auch noch toll weiterer Text}
\label{fig:sinnvolles_label3}
\end{figure}
\Blindtext

\begin{figure}
\includegraphics[scale=1]{Pictures/zweiDrittel}
\caption{Hier trauen wir uns, auch mal was dicker aufzutragen}
\label{fig:sinnvolles_label3}
\end{figure}

\begin{figure}
\includegraphics[scale=1]{Pictures/ganzseitig}
\caption{Manche Bilder verdienen auch eine ganze Seite. Diese hier eigentlich nicht.}
\label{fig:sinnvolles_label3}
\end{figure}

\Blindtext