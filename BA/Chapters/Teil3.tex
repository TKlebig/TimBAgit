\chapter{Versuchsdurchführung}
\label{cha:Versuchsdurchführung}

\section{Einstellung von Normbedingungen}
\label{sec:Einstellung von Normbedingungen}

Vor Durchführung der einzelnen Untersuchungen ist es notwendig zu prüfen ob die Umgebungsbedingungen konstant sind und sich der Norm (siehe Abschnitt~\ref{sec:Testbedingungen nach Norm}) orientieren, um reproduzierbare Ergebnisse zu erzielen. Zu diesem Zweck werden entsprechend Abbildung~\refr{fig:Messpunkte} Sensoren positioniert, die Temperatur und relative Luftfeuchtigkeit aufzeichnen. 

\section{Kltemittelöle}
\label{sec:Kältemitteloele}

\section{Abtauintervalle}
\label{sec:Abtauintervalle}

\section{Verdichter}
\label{sec:Verdichter}

\section{Änderung der Verdampferschaltung}
\label{sec:Änderung der Verdampferschaltung}















Aus den Versuchsdaten ist ersichtlich, dass nur die letzte Rohrstrecke überhitzt ist.

\begin{table}[]
\centering
\caption{Werte der Wärmeübergangs- und Wärmeleitzahlen}
\label{tab:Werte der Wärmeübergangs- und Wärmeleitzahlen}
\renewcommand{\arraystretch}{1.2}
\begin{tabular}{|l|l|l|}

\hline
Parameter        & Wert  & Einheit           \\ \hline
$\alpha_{L}$     & 292   & $\frac{W}{m^2 K}$ \\
$\alpha_{Km}$    & 50000 & $\frac{W}{m^2 K}$ \\
$\alpha_{Km,SH}$ & 29    & $\frac{W}{m^2 K}$ \\
$\lambda_{Cu}$   & 380   & $\frac{W}{m K}$   \\
$\lambda_{Al}$   & 220   & $\frac{W}{m K}$   \\ \hline
\end{tabular}
\end{table}